%Display code in article
\usepackage{listings}
\usepackage{textcomp}
\usepackage{xcolor}
 \lstset{ 
  backgroundcolor=\color{white},  % background color; e.g., nearwhite you must add \usepackage{color} or \usepackage{xcolor}; should come as last argument
  basicstyle=\footnotesize\ttfamily,% the size of the fonts that are used for the code
  breakatwhitespace=false,% sets if automatic breaks should only happen at whitespace
  breaklines=true, % sets automatic line breaking
  framextopmargin=5pt,
  framexleftmargin=5pt, 
  framexbottommargin=5pt,
  framexrightmargin=0pt,
  framesep=0pt,
  captionpos=b,% sets the caption-position to bottom
  commentstyle=\color{mygreen}, % comment style
  morecomment=[s]{/*}{*/},
  deletekeywords={...}, % if want to delete keywords from the given language
  escapeinside={\%*}{*)},% if you want to add LaTeX within your code
  extendedchars=true,% lets you use non-ASCII characters; for 8-bits encodings only, does not work with UTF-8
  frame=single,	% adds a frame around the code
  keepspaces=false, % keeps spaces in text, useful for keeping indentation of code (possibly needs columns=flexible)
  keywordstyle=\color{blue},% keyword style blue
  language=java, % the language of the code
  morekeywords={},% if you want to add more keywords to the set
  numbers=none, % where line-numbers put; possible values are (none, left, right)
  numbersep=0pt,% how far the line-numbers are from the code
  numberstyle=\tiny\color{mygray}, % the style that is used for the line-numbers
  rulecolor=\color{gray},  % appleGray     % if not set, the frame-color may be changed on line-breaks within not-black text (e.g. comments (green here))
  sensitive=true,
  showspaces=false, % show spaces everywhere adding particular underscores; it overrides 'showstringspaces'
  showstringspaces=false,% underline spaces within strings only
  showtabs=false, % show tabs within strings adding particular underscores
  stepnumber=2,	% the step between two line-numbers. If it's 1, each line will be numbered
  stringstyle=\color{purple}, % string literal style
  tabsize=4,% sets default tabsize to 2 spaces
  title=\lstname, % show the filename of files included with \lstinputlisting; also try caption instead of title
  upquote=true,      % Straight quotes
  belowcaptionskip=0em,
  belowskip=0em
}
